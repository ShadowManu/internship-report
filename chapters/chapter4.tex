\chapter{Desarrollo}

En el presente capítulo se describirán los pasos y procedimientos realizados para cumplir con los objetivos propuestos para la pasantía y se encuentra dividido en 5 secciones. Cada una representa mejoras incrementales que enriquecieron tanto los aspectos de arquitectura de software como conceptos asociados a las \pwas. Así mismo, cada etapa fue realizada de forma exploratoria, utilizando herramientas y enfoques de desarrollo retroalimentados por los resultdos parciales que eran obtenidos.

\section{Migración a Angular 2}

La aplicación web cliente de \business, fue desarrollada inicialmente con el \textit{framework} web de desarrollo AngularJS, en su versión 1. Esta herramienta permitió al equipo de desarrollo desplegar funcionalidades de forma rápida y basada en prototipos. Sin embargo, a medida que el número de requisitos que debe cumplir la aplicación aumenta, y consecuentamente la complejidad de la misma, se require hacer uso de herramientas que permitan la escalabilidad del desarrollo con el uso de tecnologías en el estado del arte que permitan la mantenibilidad y extensión del software a mayor escala.

Desde la comunidad de código libre, y con el soporte de Google para su desarrollo, se creó la versión 2 del \textit{framework} `Angular', compartiendo conceptos con su versión anterior pero con la capacidad de atacar muchas más aristas del desarrollo web. Para el momento del desarrollo de la pasantía, aún se encontraba en estado de preproducción, pero prometiendo ser una herramienta excelente para cumplir con los requisitos de escalabilidad.

Durante esta primera etapa, el equipo de Desarrollo, bajo la coordinación del pasante, enfocó buena parte de sus recursos en actualizar la aplicación existente bajo la nueva herramienta. Este proceso se realizó de forma incremental: `Angular 2' tuvo herramientas asociadas que permitían interlazar componentes de ambas versiones, de manera que nuevas funcionalidades eran agregadas bajo la versión 2, y paulatinamente se fueron migrando componentes pre-existentes de la versión 1. Luego de integrar el uso de ambas herramientas en la platforma y ajustarlo al sistema de construcción de la aplicación (\textit{builds}), esto permitió seguir manteniendo un ritmo de lanzamiento de funcionalidades aceptable mientras el proceso de actualización era realizado paralelamente. De la misma forma, todo el equipo de desarrollo tuvo la oportunidad de familiarizarse con el nuevo \textit{framework} al implementar o migrar funcionalidades en la plataforma.

\section{Capa de Comunicaciones}

COMMS LAYER (IMAGE UPLOAD)

AMAZON CF / BUILD / DEPLOYMENT

SERVICE WORKER / CACHE API

POUCHDB / INDEXEDDB
