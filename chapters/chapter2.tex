\chapter{Marco Teórico}

En el presente capítulo se describen los conceptos teóricos primordiales para la realización exitosa del proyecto. Se comienza detallando los mecanismos de comunicación utilizados por los exploradores web y tecnologías modernas asociadas a la plataforma web (colocando énfasis en los sistemas disponibles de caché y manejo de datos), para finalizar con la arquitectura del software, concepto de suma importancia para el desarrollo de software en general y en particular para encaminar el desarollo de \pwas.

\section{HTTP}

\textit{HTTP} es un protocolo de comunicación fundamental a nivel de aplicación para la transferencia de datos a través de Internet, desarrollado por la IETF. Junto a su variante cifrada \textit{https}, son el protocolo por defecto para solicitar información a servidores web. Es un protocolo basado en solicitud-respuesta para modelos de red cliente-servidor.

Superficialmente, una solicitud (\textit{request}) HTTP está conformada por: un método, referido algunas veces como el verbo, el cual indica la acción a realizar; una dirección (URL) que identifica al recurso bajo una representación textual; unos ``encabezados http'' (\textit{headers}); y una opcional carga útil (\textit{payload}), la cual contiene información extra asociada a la solicitud. Por su parte, una respuesta (\textit{response}) está conformada por \textit{headers}, \textit{payload} (opcional) y un código de estado (\textit{status code}), el cual es un número que indica el estado de la respuesta.

Los exploradores web utilizan principalmente este protocolo para solicitar los datos requeridos para presentar una página web y transferir datos entre un servidor y el explorador. Los diferentes componentes expuestos son utilizados para variados objetivos como: solicitar páginas web, autenticación de usuarios, envío de formularios, seguimiento de usuarios, sistemas de caché, etc.

\section{Caché HTTP}

El almacenamiento en caché de \textit{http} es una técnica para acelerar el desempeño de \textit{requests}, aprovechando recursos que han sido solicitados previamente y se consideran localmente válidos. Existen varios tipos de almacenes de caché HTTP. Las dos categorías más importantes son la caché privada del explorador y la caché ``proxy'' compartida. La primera es manejada internamente por el explorador en base a respuestas recibidas, mientras que la segunda puede estar implantada en diferentes lugares de la red, tal como el proveedor del servicio de red-internet (ISP) y el proveedor del servidor web. Generalmente, implantar un sistema de caché HTTP consiste en modificar las respuestas salientes del servidor web, para dirigir en la medida de lo posible el comportamiento de la caché del explorador web.

Hay diferentes formas de implantar un sistema de Caché HTTP en un servidor web, y cada una permite manejar diferentes casos de uso. Durante el presente trabajo, se considerará la técnica de almacenamiento en caché basada en el encabezado HTTP llamado \texttt{Cache-Control}.

El encabezado \texttt{Cache-Control} consiste en un conjunto de directivas que especifican el comportamiento de los sistemas de caché. Estas directivas especifican si la información es pública o privada (si los intermediarios deben guardarla o no respectivamente), por cuánto tiempo es válida, si deben revalidar con una caché compartida antes de utilizar un recurso almacenado en caché, etc. Otros encabezados asociados complementan a estas directivas, como el encabezado de respuestas \texttt{ETag}, el cual consiste en un identificador el cual el explorador utiliza para confirmar que un recurso en caché sigue siendo válido y utilizable. Dependiendo del caso de uso necesario, se utiliza una combinación de directivas apropiadas al recurso a mantener en el sistema de caché.

\section{Service Worker}

\textit{Service Worker} es una tarea de fondo dirigida por eventos el cual sirve como intermediario (\textit{proxy}) de solicitudes salientes del explorador. Es implementada en JavaScript por el desarrollador de la página web en la cual el \textit{service worker} se registra y le permite tener una herramienta de bajo nivel con la cual puede interceptar una solicitud saliente del explorador y decidir cómo manejarlo. Es una tecnología bastante moderna y entre algunas de sus usos posibles, se encuentra preparar la aplicación web para funcionar sin conexión (\textit{offline}), administrar el almacenamiento de caché local, redirigir y cancelar solicitudes en ciertas condiciones, etc. Es un elemento esencial para la creación de applicaciones resilientes a entornos de baja conectividad, como lo son los dispositivos móviles, que pueden encontrarse comúnmente con una conexíón a Internet con calidad intermitente.

\section{Cache API}

La interfaz de programación de \texttt{Cache} es una tecnología moderna disponible en exploradores modernos y funciona como mecanismo de almacenamiento local de parejas de solicitudes y respuestas. Si bien es utilizable de forma independiente, está ajustada para ser usada en la implementación del \textit{service worker}, para especificar flujos de almacenamiento en caché personalizados.

\section{Angular Framework}

Angular es un ``entorno de trabajo'' (\textit{framework}) de código libre desarrollado por Google, orientado a simplificar el desarrollo de aplicaciones web ``RIA'' y ``SPA''. Se define un ``entorno de trabajo'' como un conjunto cohesivo de librerías de software que ofrecen utilidades y herramientas para simplificar y unificar el desarrollo de un producto de software. En el caso particular de Angular, es un ``entorno de trabajo'' en el lenguaje ``TypeScript'', disponible para crear aplicaciones web para ser ejecutadas en exploradores web modernos. Durante la presentación de este trabajo, se hará mención de este proyecto de software en su versión 1 (mencionada como ``AngularJS'') y su versión 2 en adelante (mencionada simplemente como ``Angular''). Si bien ambas versiones tienen objetivos muy cercanos, Angular introduce una serie de conceptos, implementaciones y decisiones arquitectónicas que incrementan la capacidad de los equipos de desarrollo de proveer nuevas funcionalidades y mejoras, siguiendo prácticas e innovaciones industriales originadas de variadas fuentes de las comunidades de código libre y empresas comerciales asociadas.

\section{Observables}

Los observables son un tipo abstracto de datos que modela la idea de valores a través del tiempo. Posee sus raíces tanto en Programación Funcional Reactiva (FRP) y sus modelos, como en patrones de diseño de software, en particular el patrón Observador, el patrón Iterador, y el paradigma de programación funcional. ``ReactiveX'' constituye la implementación y especificación de tal modelo, el cual ha sido implementado en variados lenguajes (Java, C\#, Swift, Javascript, etc). Esta librería permite implementar bajo modelos generales sistemas que requieran reaccionar a múltiples eventos de formas no triviales. Este modelo es agnóstico del ambiente en que se utiliza lo cual permite adaptarlo en diferentes escenarios y casos de uso. Angular hace considerable uso de esta abstracción, reflejada en la implementación ``rxjs'', dirigida a la plataforma Javascript.

\section{Typescript}

``Typescript'' es un lenguaje de programación de código libre desarrollado por Microsoft, diseñado para el desarrollo y mantenibilidad de aplicaciones web. De forma superficial, el lenguaje ofrece un sistema de tipos de programación para revisión e inferencia a tiempo de compilación, soporte a funcionalidades de versiones de Javascript modernas en sistemas tradicionales y otros artefactos de ingeniería de software. Es un lenguaje compilado de \textit{fuente-a-fuente}, en el cual el resultado final es código Javascript, para ser ejecutado en exploradores web u otros ambientes que ejecuten código fuente Javascript.

\section{SPA - Single Page Application}

El término acuñado ``SPA'' (aplicación de página única)' describe al conjunto de portales web donde, a través de diversas técnicas de programación, el portal no consiste en un conjunto de documentos relacionados a través de hipervínculos (navegación tradicional en navegadores web), sino en una página única que dinámicamente modifica el contenido de la misma, imitando la experiencia de aplicaciones digitales clásicas pero dentro de un ambiente web.

\section{PWA - Progressive Web Application}

El término acuñado ``PWA'' (aplicación web progresiva) se refieren a aquellas aplicaciones (generalmente de página única), que de forma progresiva poseen funcionalidades que sólo han prevalecido en aplicaciones tradicionales instaladas en los dispositivos, pero que mantienen (e intensifican) las ventajas de la plataforma. Estas aplicaciones implementan aspectos como: funcionalidad sin acceso a la red internet, acceso al \textit{hardware} dispositivo, notificaciones, comunicaciones nativas, etc; tomando ventaja de las nuevas características disponibles en exploradores web modernos y ofreciendo experiencias cercanas (o mejores) a las plataformas tradicionales de aplicaciones.
