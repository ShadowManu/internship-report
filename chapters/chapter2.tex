\chapter{Marco Teórico}

En el presente capítulo se describen los conceptos teóricos primordiales para la realización exitosa del proyecto. Se comienza detallando los mecanismos de comunicación utilizados por los exploradores web y tecnologías modernas asociadas a la plataforma web (colocando énfasis en los sistemas disponibles de caché y manejo de datos), para finalizar con la arquitectura del software, concepto de suma importancia para el desarrollo de software en general y en particular para encaminar el desarollo de \pwas.

\section{HTTP}

\textit{HTTP} es un protocolo de comunicación.

% BROWSER HTTP COMMS
% HTTP CACHE
% SERVICE WORKER

% ANGULAR FRAMEWORK
% OBSERVABLES
% RXREST

% SPA
% PWA

% ARCHITECTURE
%   APP
%   DEPLOY

% IMAGE HANDLING
