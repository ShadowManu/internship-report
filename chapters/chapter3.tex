\chapter{Marco Metodológico}

En este capítulo se explica la metodologíia de trabajo utilizada por \business para realizar sus proyectos. Se detalla cual es su filosofía, características, principios, fundamentos y diferencias con otra metodología similar.

Los desarrolladores dentro de la empresa utilizan la metodología \textit{Continuous Delivery}, la cual es una metodología de la filosofía \textit{Lean}.

\textit{Continuous Delivery} es la habilidad de pasar todos los cambios de todos los tipos a producción, o a la mano de ls usuarios de una manera segura, rápida y sostenible. Para lograr esta meta, el código siempre debe estar listo para su despliegue.

Entre las ventajas que tiene esta metodología se destaca:

\begin{itemize}
    \item Liberaciones menos riesgosas
    \item Menor tiempo para lanzar al mercado
    \item Mejor calidad
    \item Costos menores
    \item Mejores productos
    \item Equipos de trbajo más felices
\end{itemize}

\section{Principios del Continuous Delivery}

\textit{Continuous Delivery} posee cinco principios:

\begin{enumerate}
    \item Construcciones de calidad: Eliminar la necesidad de revisión manual en los productos para garantizar la calidad del mismo. Es mucho mós barato arreglar los problemas si si encuentran inmediatamente mediante el uso de pruebas automáticas.

    \item Trabajar en lotes pequeños: Trabajar en lotes pequeños tiene múltiples ventajas, como reducción del tiempo que toma obtener retroalimentación del trabajo, lo que hace más fácil la evaluación y solución de problemas e incrementa la motivación y la eficiencia.

    \item Las computadoras realizan tareas repetitivas, las personas resuelven problemas: La meta es que las computadoras realicen tareas simples y repetitivas como pruebas de regresión, mientras que los humanos se encfocan en la solución de problemas.

    \item Búsqueda implacable de la mejora continua: \textit{kaizen} en japonés, idea tomada del movimiento \textit{Lean}, plantea que las mejores organizaciones son aquellas donde todos tratan de mejorar como una parte esencial de su trabajo, y nadie está satisfecho con el status quo.

    \item Todos son responsables: Todos trabajan para lograr las metas de la empresa, en lugar de que se optimice según los beneficios de un equipo o departamento.

\end{enumerate}

\section{Fundamentos del Continuous Delivery}

\textit{Continuous Delivery} posee tres fundamentos apliamente conocidos:

\begin{enumerate}

    \item Gestión de la configuración: La automatización de los despliegues, pruebas de regresiones y provisionamiento de infraestructura debe estar bajo un control de versiones, desde simples \textit{scripts} y pruebas, hasta las configuraciones y librerías necesarias para el proceso.

    \item Integración continua: Los desarrolladores integran su trabajo regularmente y se ejecutan una serie de pruebas de regresión antes y después para evitar errores.

    \item Pruebas continuas: Se debe realizar pruebas continuamente para asegurar un producto de calidad, tanto si son manuales como automáticas según sea necesario.

\end{enumerate}

Es de destacar la diferencia entre \textit{continuous delivery} y \textit{continuous deployment}, y es que en este último, cada cambio que pasa las pruebas automáticas es desplegado en producción. Mientras que en el primero, sólo se garantiza que en cualquier momento se pueda hacer el despliegue a producción, por la cual éste no es automático.

Además de \textit{Continuos Delivery}, también se toman ideas de otras metodologías de desarrollo ógil como \textit{Scrum}, por ejemplo el uso de \textit{sprint} de desarrollo de 2 semanas o el uso de tablero \textit{Kanban} para el ordenamiento de las tareas por hacer.
