\addcontentsline{toc}{chapter}{Lista de Abreviaturas}

\begin{center}
\textsc{\bfseries\uppercase{Lista de Abreviaturas}}
\end{center}

{
\setlength{\parskip}{1em}

\textbf{AWS} \textit{Amazon Web Services} - Servicios Web de Amazon

\textbf{CDN} \textit{Content Delivery Network} - Red de Distribución de Contenidos

\textbf{FRP} \textit{Functional Reactive Programming} - Programación Funcional Reactiva

\textbf{HTTP} \textit{HyperText Transfer Protocol} - Protocolo de Transferencia de Hipertexto

\textbf{HTTPS} \textit{HTTPS over SSL}

\textbf{IndexedDB} \textit{Indexed Database}

\textbf{IETF} \textit{Internet Engineering Task Force} - Comité de Ingeniería de Internet

\textbf{ISP} \textit{Internet Service Provider} - Proveedor de Servicio de Internet

\textbf{PouchDB} \textit{Pouch Database}

\textbf{PWA} \textit{Progressive Web App} - Aplicación Web Progresiva

\textbf{RESTful} \textit{Representational State Transfer}

\textbf{RIA} \textit{Rich Internet Application} - Aplicación de Internet Rica en Contenidos

\textbf{RXJS} \textit{Reactive Extensions for JavaScript} - Extensiones Reactivas para Javascript

\textbf{SPA} \textit{Single Page Application} - Aplicación de Página Única
}

\newpage
