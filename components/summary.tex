\setcounter{page}{4}
\begin{center}
\textsc{\bfseries\uppercase{Resumen}}
\end{center}

La plataforma web de \business ofrece una serie de funcionalidades dirigidas a ser utilizadas a través de dispositvos móviles. Principalmente en estos dispositivos, la conexión a internet tiende a ser restringida o nula. Aún bajo estas circunstancias, los usuarios desean recibir experiencias de alta calidad e interactividad, que puedan responder antes estas fallas y funcionar de manera apropiada como con aplicaciones nativas instaladas en el dispositivo.

Por otro lado, con el crecimiento de la plataforma y la inserción de nuevas funcionalidades y requisitos, la complejidad del software incrementaron la dificultad de realizar modificaciones y mejoras sobre la plataforma. Igualmente, la velocidad en diferentes aspectos de la aplicación afectaban de forma considerable la experiencia de usuario y sus expectativas.

En el presente informe, se describen las actividades realizadas durante el proyecto de pasantía, que consistieron en mejorar la arquitectura de software de la plataforma web de \business, diseñar e implementar un nuevo sistema de comunicacion ajustado a los requerimientos de la plataforma, resiliente a conexiones restringidas y escalable,e implementar los sistemas que permitan el inicio y uso de la aplicación con conexiones restringidas.

Para la realización de la pasantía fue necesario un estudio de la arquitectura actual de \business, y una investigación suficiente sobre el estado del arte en cuanto a tecnologías y herramientas disponibles para cumplir con los objetivos planteados, así como sus ventajas y limitaciones.

\newpage

