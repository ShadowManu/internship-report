\addcontentsline{toc}{chapter}{Introducción}

\begin{center}
\textsc{\bfseries\uppercase{Introducción}}
\end{center}

En la actualidad, el uso de tecnologías de la información a través de dispositivos móviles crece a un ritmo acelerado. De forma colateral, el número de emprendimientos comerciales que buscan satisfacer diferentes necesidades también se encuentra en alza.

En \business, se planteó la creación de una plataforma en donde personas puedan publicar actividades y otras pueden reservarlas y asistir a ellas, buscando reunir a profesionales independientes que realizan actividades que les apasiona con aquellas que les gustaría realizarlas, en un entorno seguro y confiable. El crecimiento de la plataforma orientada a investigar su viabilidad comercial y utilidad, ciertos aspectos técnicos de la ingeniería de software de la plataforma web quedaron relegados a segundo plano, como la reducción de deuda técnica y optimizaciones no funcionales.

Este proyecto de pasantía busca solventar estos aspectos al realizar un análisis exhaustivo de la arquitectura actual de su plataforma web así como las mejores prácticas y herramientas del estado del arte, de manera que se incrementa la productividad del equipo de desarrollo, la velocidad de la plataforma, la escalabilidad del sistema, y la experiencia del usuario, en particular, con respecto al uso de la plataforma en condiciones de conectividad restringida.

Consecuentemente, los objetivos principales de la pasantía son:

\begin{itemize}
  \item Analizar de la arquitectura actual de la plataforma web de \business.
  \item Investigar de las buenas prácticas en el estado del arte en relación a la arquitectura de software web, principalmente con respecto a escenarios con conexión restringida o nula.
  \item Diseñar e implementar las reestructuraciones y sistemas en base a las herramientas disponibles que faciliten el desarrollo continuo de nuevas funcionalidades y sistemas de resiliencia a conexiones restringidas
\end{itemize}

Este informe está estructurado de la siguiente manera: el capítulo 1 describe el entorno empresarial en el que fue desarrollado el proyecto. El capítulo 2 define conceptos teóricos importantes para la pasantía, así como herramientas y tecnologías asociadas. El capítulo 3 indica las técnicas y métodos usados durante el desarrollo. El capítulo 4 detalla el proceso de desarrollo del proyecto. Por último, se presentan las conclusiones y recomendaciones.
